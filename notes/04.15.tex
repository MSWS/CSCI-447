\documentclass{article}

\begin{document}
\section*{Process Creation / Termination}

\subsection*{Termination}

\begin{itemize}
    \item Process issues an exit instruction
\end{itemize}

\subsection*{Cooperating processes need to communicate}
\textbf{Why?}
\begin{itemize}
    \item Sharing data
    \item Computational Speedup
    \item Modularity
\end{itemize}

\begin{description}
    \item[Information Sharing] two+ processes share the same piece of info
    \item[Computational Speedup]
    \item[Modularity]
\end{description}

\subsubsection*{Examples}
\begin{itemize}
    \item Producer / Consumer Model
\end{itemize}

\begin{verbatim}
    co
    // - Parrelization
    oc
\end{verbatim}

When referencing concurrency, we use the term `arms' to describe the number of
processes. A program can have two or MORE arms.

\textbf{Independent}: `Can operate without each other, and complete its task w/o
intervention of something else getting in the way'

\begin{verbatim}
    string line;
    read a line from stdin into line;
    while( !EOF) {
        co look for pattern in line;
            if (pattern is in line)
                write lin;
        // read next line of input into line
        oc;
    }
\end{verbatim}

The above example is \underline{not} independent, as they both use the line buffer.

\begin{verbatim}
    string line1, line2;
    read a line from stdin into line1;
    while( !EOF) {
        co look for pattern in line1;
            if (pattern is in line1)
                write line1;
        // read next line of input into line2
        oc;
        line1 = line2;
    }
\end{verbatim}
What did we change?
\begin{itemize}
    \item At the end of each loop iteration, copy the contents of \texttt{line2} into \texttt{line1}.
\end{itemize}

Is this solution efficient?

At each iteration of the while loop, how many concurrent processes are created, completed, and destroyed?
2

How do you determine if two processes are independent?

\begin{description}
    \item[Read] set of variables read by a process
    \item[Write] set of variables written by a process
\end{description}

Separate processes \(P_a, P_b\) on separate CPUs

A process/thread may have a private set of variables scoped only to that thread.

Two processes are \textbf{independent} if the write set of each is disjoint from
both the read and write sets of the other.

\begin{itemize}
    \item Independence
    \item Threads
    \item Synchronization
\end{itemize}

\end{document}