\documentclass{article}

\usepackage{amsmath, amssymb}

\begin{document}
\section*{System Calls}
Think of a system call as a request by a user for the system.
\subsection*{Several `Basic' Types}
\begin{enumerate}
      \item Process Control \\
            Need to halt, normally or via abort
      \item File Management (create, move, etc. files) \\
            Manage files and directories
      \item Device Management \\
            Process might need multiple resources
      \item Information Management \\
            Keep track of up-time and waiting
      \item Communications - I want to send a message \\
            Message passing and shared memory
      \item Protections Protect read, allow modification by owner only \\
            Manipulate protections of resources
\end{enumerate}

\section*{Processes}
\subsection*{States of Processes}
\begin{description}
      \item [New:] being created
      \item [Running:] being executed
      \item [Waiting:] for some event
      \item [Ready:] waiting to be assigned a processor
      \item [Terminated:] finished executing
\end{description}

We can use these states for a state diagram.

\subsection*{Process Control Block}
A `snapshot' of a process, and where it is in its execution lifecycle.
\begin{itemize}
      \item State
      \item Program Counter
      \item CPU Registers
      \item CPU Scheduling Info
      \item Memory Management Info
      \item Accounting Info
      \item I/O Status Info
\end{itemize}

Typically PCB lists are implemented as a queue.
\end{document}