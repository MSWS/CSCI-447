\documentclass{article}

\usepackage{amsmath, amssymb}
\begin{document}

More work needed to calculate \texttt{base + limit} vs checking just the base.

Thus, we check if a given address is \(\geq\) the base and THEN
we check if it is \(\leq\) the limit.

Base checked first, then base + limit violation
\begin{align*}
	q: & 3 ops         & 19/3 \text{ avg} \\
	r: & 3 + 5 = 8 ops                    \\
	s: & 3 + 5 = 8 ops
\end{align*}

Base + limit checked first, then base violation
\begin{align*}
	q: & 5 ops         & 21/3 \text{ avg} \\
	r: & 5 + 3 = 8 ops                    \\
	s: & 5 + 3 = 8 ops
\end{align*}

When in memory should a process be placed?
\begin{verbatim}
  for (int x = 0; x < 10; x++) {
    string s = input() // prompt user
  }
\end{verbatim}

The OS knows the size of an int, but it doesn't know the size of the input.

\begin{quote}
  IF you know the size of the datum, and where in memory it might reside at compile time, then reserve the memory at compile time.
\end{quote}

What is the downside of this approach?

\begin{itemize}
  \item Overflow
  \item Not running
  \item Coordination
\end{itemize}
\end{document}
