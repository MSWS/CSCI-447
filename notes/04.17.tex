\documentclass{article}

\usepackage{amsmath, amssymb}

\begin{document}
\begin{description}
    \item[Concurrent: ] Two or more processes are making progress
    \item[Parallel: ] Two or more things are actively running at the same time
\end{description}

With m threads executing n atomic actions each, there are 6 possible histories

\(_3 P _3\)

What about 3 threads executing 2 actions each?

\begin{equation*}
    < 6!
\end{equation*}

90

2 threads executing 3 actions each?

Is there a formula for this?

\begin{equation*}
    \frac{(mn)!}{(n!)^m}
\end{equation*}

Concurrent may have wait time, parallel does not

On a single CPU, concurrency is possible, but not parallelism

On a multiple CPU system, parallelism is possible

\textbf{What happens when a thread is killed mid-execution?}

Cancellations

\begin{description}
    \item[Asynchronous: ] The target thread is immediately terminated
    \item[Synchronous: ] The target thread periodically `checks in' to find out if it should be terminated
\end{description}


\end{document}