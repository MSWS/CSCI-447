\documentclass{article}

\usepackage{amsmath, amssymb}
\usepackage{geometry}[margin=1in]
\usepackage{enumitem}

\author{Isaac Boaz}
\title{CSCI 447 Homework 3}

\begin{document}
\maketitle

\subsection*{Question 1}
Having a very small quantum requires the PC to do a lot more work that isn't
directly executing processes, whether that be
\begin{itemize}
    \item Context switching between processes
    \item Decoding processes
    \item Scheduling processes
\end{itemize}

While a lower quantum may ensure more processes get a more fair distribution of
time, it comes at the sacrifice of CPU efficiency.

\subsection*{Question 2}
\begin{itemize}
    \item If FCFS is used:
          \begin{tabular}{ccccc}
              Process Order   & \(P_1\) & \(P_2\) & \(P_3\) & Average \\
              Arrival Time    & 0       & 0.4     & 1.0               \\
              Execution Time  & 0       & 8       & 12                \\
              Turnaround Time & 8       & 11.6    & 12      & 10.53   \\
          \end{tabular}
    \item If SJF is used:
          \begin{tabular}{ccccc}
              Process      Order & \(P_1\) & \(P_3\) & \(P_2\) & Average \\
              Arrival Time       & 0       & 1       & 0.4               \\
              Execution Time     & 0       & 8       & 9                 \\
              Turnaround Time    & 8       & 8       & 12.6    & 9.53    \\
          \end{tabular}
    \item If SJF is used: with idle from 0 to 1
          \begin{tabular}{ccccc}
              Process      Order & \(P_1\) & \(P_3\) & \(P_2\) & Average \\
              Arrival Time       & 0       & 1       & 0.4               \\
              Execution Time     & 1       & 9       & 10                \\
              Turnaround Time    & 9       & 9       & 13.6    & 10.53   \\
          \end{tabular}
\end{itemize}

\subsection*{Question 3}
\begin{enumerate}[label=\alph*.]
    \item First-come, first-served could result in starvation if a process with
          a long execution time is placed at the front of the queue.

          Eg: \(P_1\) has an execution time of 100, and \(P_2\) has an execution
          time of 1. If \(P_1\) is placed at the front of the queue, \(P_2\)
          will never be executed.
    \item Round robin most likely will not result in starvation, as each process
          is given a quantum of time to execute. However, if the quantum is too
          small, the CPU could spend more time context switching than executing
          processes.
    \item Shortest Job First could still result in starvation depending on the
          order that jobs arrive to the CPU, and the length of the jobs. If a long job
          arrives first, and a series of short jobs arrive after, the short jobs may
          never be executed.
    \item Priority scheduling could result in starvation if a process with a high
          priority is placed at the front of the queue, and never completes.
\end{enumerate}

\subsection*{Question 4}
\begin{enumerate}[label=\alph*.]
    \item 430 + 219 = 649
    \item 400 + 1327 = 1727
    \item 10 + 2300 = 2310
    \item 112 + 1952 = 2064
    \item 500 + 90 = 590
\end{enumerate}
\end{document}