\documentclass{article}

\usepackage{amsmath, amssymb}
\usepackage{enumitem}
\usepackage{geometry}[margin=1in]
\title{\vspace{-1em}CSCI 447 Homework 1}
\author{Isaac Boaz}

\begin{document}
\maketitle

\begin{enumerate}
    \item The three choices that best specify the main functional roles of an OS are
          \begin{itemize}
              \item Allocate Resources
              \item Control and Supervise the execution of multiple processes
              \item Act as an intermediate between the user and hardware
          \end{itemize}
    \item One example of when an OS might want to \textit{permit the CPU to be
              idle} even when some processes are in need of compute resources are when the
          device that the OS is running on has a limited battery, and thus the CPU is
          idle to conserve power.
    \item The following instructions should be run in kernel mode.
          \begin{itemize}
              \item Clear Memory
              \item Modify the entries in a device-status table
              \item Turn off interrupts
              \item Switch from user to kernel mode
          \end{itemize}
    \item While such a computer could potentially be powerful, having a computer
          that is only cache would disallow it to have persistent storage. As a
          result, the computer would need to be repeatedly programmed every time it
          shut down.
    \item Much like how \texttt{sudo} is limited to those with the proper
          permissions, and by default is not enabled, restricting access to processes,
          memory, and other sensitive / private data to the kernel requires
          applications go through the OS or use system calls. This allows for
          moderating, auditing, and authorized access of these kernel-level actions.
    \item Interrupts inform the OS that a device requires attention. Traps are a type
          of interrupt that are triggered by software. System calls are a type of trap
          that are used to request services from the OS.
    \item Ideally DMA should have no effect on the CPU's operation, as it uses a
          separate bus to communicate with memory. In the instances where both the DMA
          and CPU are trying to access the same memory (level?), there may be overhead to
          ensure that the CPU and DMA do not interfere with each other.
    \item Some challenges that arise when developing an OS for a mobile device are:
          \begin{itemize}
              \item Balancing power consumption with performance
              \item Privacy and Security in addition to compatability
              \item Supporting many different applications and support for
                    background tasks
          \end{itemize}
\end{enumerate}

\end{document}