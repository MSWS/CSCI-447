\documentclass{article}

\usepackage{enumitem}

\begin{document}

\section*{Question 1}
None of the disk-scheduling disciplines except FCFS prevents starvation.

\begin{enumerate}[label=\alph*.]
    \item This is true because every other disk-scheduling discipline does not
          take into account how old a request is. Thus, it is always possible to
          prevent a given request by inserting a `higher-priority' request.
    \item One possible way to modify any algorithm to prevent starvation is to
          have them take into account the age / time since the request was made, and
          grant higher priority to those that are very old.
    \item Fairness is important for time-sharing systems as the OS does not
          always know what processes have what priority, or what the user
          exactly needs done immediately / in a timely fashion, so it is
          important to be fair to all processes. One example of fairness being
          important is if a user wants to load up some files from an SD card
          while at the same time download a file from the internet, the OS
          should be able to both read the SD card and save the file in a
          reasonable time.
    \item \begin{itemize}
              \item If the computer is about to shut down, it might ignore new
                    I/O requests and prioritize critical applications / kernel-level
                    I/O requests.
              \item If a process has a very low priority (eg. a background task
                    that takes 10+ hours to do), then the OS might ignore it in favor
                    of more important tasks.
              \item If a user is actively using a process, the OS might
                    prioritize that process over others.
          \end{itemize}
\end{enumerate}

\end{document}