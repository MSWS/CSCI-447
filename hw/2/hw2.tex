\documentclass{article}

\usepackage{amsmath, amssymb}
\usepackage{listings}
\usepackage{geometry}[margin=1in]
\usepackage{enumitem}
\usepackage[utf8]{inputenc}

\title{CSCI 447 Homework 2}
\author{Isaac Boaz}

\begin{document}
\section*{Question 1}

\begin{lstlisting}[language=Python, showstringspaces=false]
import itertools

tasks = ['a1', 'a2', 'b1', 'b2', 'b3', 'c1', 'c2', 'c3', 'd1', 'd2']
    
def main():
    perms = itertools.permutations(tasks)
    perms = filter(is_in_order, perms)
    permList = list(perms)
    valid_order = len(permList)
    print("Ordered permutations: ", valid_order)
    print(permList[0]) 
    permList = list(filter(is_valid, permList))
    print(permList[0]) 
    valid_constraint = len(permList)
    print("Valid permutations: ", valid_constraint)
    
    print(valid_constraint / valid_order)
    
def is_in_order(perm: list[str]) -> bool:
    if perm.index('a1') > perm.index('a2'):
        return False
    if perm.index('b1') > perm.index('b2') or perm.index('b2') > perm.index('b3'):
        return False
    if perm.index('c1') > perm.index('c2') or perm.index('c2') > perm.index('c3'):
        return False
    if perm.index('d1') > perm.index('d2'):
        return False
    return True
    
def is_valid(perm: list[str]) -> bool:
    if perm.index('a1') > perm.index('b3'):
        return False
    if perm.index('c2') > perm.index('a2'):
        return False
    if perm.index('d1') > perm.index('b2'):
        return False
    return True
    
if __name__ == '__main__':
    main()
    \end{lstlisting}

\subsection*{Output}
\begin{verbatim}
Ordered permutations:  25200
('a1', 'a2', 'b1', 'b2', 'b3', 'c1', 'c2', 'c3', 'd1', 'd2')
('a1', 'b1', 'c1', 'c2', 'a2', 'c3', 'd1', 'b2', 'b3', 'd2')
Valid permutations:  11144
0.44222222222222224
\end{verbatim}

Thus, with 25200 ordered permutations, only 11144 are valid. This means that 44.22\% of the ordered permutations are valid.

\section*{Question 2}
\begin{tabular}{|cc|c|cc|c|cc|}
    \multicolumn{2}{c}{Thread A} &                    & \multicolumn{2}{c}{Thread B} &     & \multicolumn{2}{c}{Thread C}                               \\
    \hline
    IA1                          & \texttt{x = x + 1} &                              & IB2 & \texttt{x = x - 2}           &  & IC1 & \texttt{x = x + 2} \\
    IA2                          & \texttt{print(x)}  &                              & IB2 & \texttt{print(x)}            &  & IC2 & \texttt{print(x)}  \\
    \hline
\end{tabular}

\begin{tabular}{|cc|c|cc|c|cc|}
    \multicolumn{2}{c}{Thread A} &                   & \multicolumn{2}{c}{Thread B} &        & \multicolumn{2}{c}{Thread C}                                 \\
    \hline
    IA1\_1                       & \texttt{y <- x}   &                              & IB2\_1 & \texttt{y <- x}              &  & IC1\_1 & \texttt{y <- x}   \\
    IA1\_2                       & \texttt{y += 1}   &                              & IB2\_2 & \texttt{y -= 2}              &  & IC1\_2 & \texttt{y += 2}   \\
    IA1\_3                       & \texttt{x <- y}   &                              & IB2\_3 & \texttt{x <- y}              &  & IC1\_3 & \texttt{x <- y}   \\
    IA2                          & \texttt{print(x)} &                              & IB2    & \texttt{print(x)}            &  & IC2    & \texttt{print(x)} \\
    \hline
\end{tabular}

A and B are impossible due to the instructions never lining up to both decrease
and then increase by 1 (or vice-versa).

% To help simplify, let's analyze the possibilities of each thread when we
% initialize \(x = 0\), including if they were to be interrupted.

% \begin{enumerate}[label=\Alph*]
%     \item Incremenets \texttt{x} by one.
%           \begin{enumerate}[label=\arabic*.]
%               \item Store x into a tmp variable (say y)
%               \item Increment y by one
%               \item Store y back into x
%           \end{enumerate}
%           Thus, the possible outcomes for if \(x\) were to be interfered with
%           while A processes are as follows:
%           \begin{enumerate}[label=\arabic*.]
%               \item If we modified \(x\) to be some value other than \(0\) subsequent
%                     to this instruction executing, then the completion of this thread
%                     could result in \(x\) being fetched as 0.
%               \item If 1 were true, then this would override the previously set
%                     \(x\) value.
%               \item Consequently this would store the new \(x\), overriding the
%                     previously set value.
%           \end{enumerate}

%           As a result, the net possibilities of this thread are either setting
%           \(x\) to 1, or \(x + 1\).
%     \item Decrements \texttt{x} by two. \\
%           Following the above steps, we see that Thread B could set \(x\) to
%           \(x\), or \(x - 2\).
% \end{enumerate}
% C (151) is possible, initialize \(x = 2\)

% \begin{tabular}{c|c|c|l}
%     Thread & Instruction & x & output \\
%     \hline
%     B      & Get 3       & 3 &        \\
%     A      & Get 3       & 3 &        \\
%     A      & 4 = 3 + 1   & 3 &        \\
%     C      & Get 3       & 3 &        \\
%     A      & Set 4       & 4 &        \\
%     B      & 1 = 3 - 2   & 4 &        \\
%     B      & Set 1       & 1 &        \\
%     B      & Print x     & 1 & 1
% \end{tabular}

All of them except D is impossible.

D (223) is possible, initialize \(x = 4\)

\begin{tabular}{c|c|c|l}
    Thread & Instruction & x & output \\
    \hline
    B      & Get 4       & 4 &        \\
    C      & Get 4       & 4 &        \\
    B      & 2 = 4 - 2   & 4 &        \\
    C      & 6 = 4 + 2   & 4 &        \\
    C      & Set 6       & 6 &        \\
    B      & Set 2       & 2 &        \\
    B      & Print x     & 2 & 2      \\
    C      & Print x     & 2 & 22     \\
    A      & Get 2       & 2 & 22     \\
    A      & 3 = 2 + 1   & 2 & 22     \\
    A      & Set 3       & 3 & 22     \\
    A      & Print x     & 3 & 223    \\
\end{tabular}

The rest are similarly possible, though I do not care enough to figure out the
correct ordering \footnote{Points for honesty though?}.

\section*{Question 3}
\subsection*{Advantages}
Simpler to construct and debug.

\subsection*{Disadvantages}
Requires thoughtful planning and design of what each layer does and what layers are needed.
Each layer adds some overhead when passing data between layers.

\section*{Question 4}
Calling 3 forks would result in 8 processes.

\section*{Question 5}
Only the Shared Memory Segments are shared.

\section*{Question 6}
For 2 processing cores with a 65\% parallel component we get a speedup of 1.48.
For 4 processing cores with a 65\% parallel component we get a speedup of 1.95.
\end{document}